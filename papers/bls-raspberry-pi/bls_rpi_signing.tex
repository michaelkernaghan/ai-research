
\documentclass[11pt]{article}
\usepackage[a4paper,margin=1in]{geometry}
\usepackage{hyperref}
\usepackage{lmodern}
\usepackage{microtype}
\usepackage{parskip}

\title{\textbf{Remote BLS Signing on Raspberry Pi:\\
A Study in Lightweight Cryptographic Infrastructure}}
\author{Dalhousie University of British Columbia\\
Department of Attestation and Ledger Studies (DALS)}
\date{August 2025}

\begin{document}
\maketitle

\begin{abstract}
This paper presents an investigation into the feasibility of deploying a
Boneh--Lynn--Shacham (BLS) remote signer on Raspberry Pi-class commodity
hardware in the context of the Tezos blockchain. The project, operated under
the auspices of the fictional \emph{Dalhousie University of British Columbia
(DUBC)}, explores whether such constrained hardware can perform remote signing
duties with sufficient reliability and security for participation in consensus.

Our work leverages the \texttt{tezos-rpi-bls-signer} implementation and targets
the Seoulnet test network, with emphasis on aggregated attestations and the
operational requirements discussed in the Tezos Agora ecosystem. Metrics include
signing latency, attest throughput, and resource utilization under varying load
conditions.

We argue that BLS-based cryptographic infrastructure need not be confined to
industrial-scale clusters. By demonstrating viability on Raspberry Pi, we show
that threshold signatures, multiparty cryptography, and future extensions can
be prototyped on low-power, widely available hardware. This accessibility
aligns with the decentralization ethos, widening the scope of participation in
advanced consensus research.
\end{abstract}

\textbf{Keywords:} Tezos; BLS Signatures; Remote Signer; Raspberry Pi; Aggregated Attestations;
Commodity Hardware; Distributed Ledger; Cryptographic Infrastructure

\section{Introduction}
The design of secure remote signers is critical to the operation of proof-of-stake
blockchains such as Tezos. Recent advances, including the introduction of BLS
signatures and aggregated attestations, impose new requirements on signing
infrastructure. While production deployments often rely on dedicated HSMs and
enterprise-class servers, the role of commodity hardware as a research and
educational platform has not been fully explored.

This paper investigates the deployment of a BLS signer on Raspberry Pi hardware.
We seek to answer whether such devices can produce reliable cryptographic
attestations in a live test network environment, and what trade-offs emerge in
terms of performance, security, and maintainability.

\section{Related Work}
The Tezos community has documented operational considerations for
aggregated attestations on public forums such as Tezos Agora
(\href{https://forum.tezosagora.org/t/aggregated-attestations-a-heads-up-for-bakers/6770}{Heads-Up for Bakers}).
In parallel, remote signing frameworks and hardware-backed approaches
(e.g., enterprise HSMs) have been evaluated for security and throughput.
Our contribution is to examine the feasibility envelope on a low-power,
readily available single-board computer while focusing specifically on
BLS signing requirements.

\section{Methodology}
We deployed \texttt{tezos-rpi-bls-signer} on a Raspberry Pi 4B with 4 GB RAM,
networked to a Seoulnet test node. Systemd hardening, firewall rules, and
monitoring were configured to simulate a responsible operator environment.
Metrics were collected on signing latency, CPU/memory utilization, and attest
success rates under both nominal and stressed conditions.

The experiment spanned multiple epochs on Seoulnet to observe signer behavior
under differing chain conditions. Configuration artifacts included a systemd
service unit, read-only root filesystem for the signer user, and IP allowlists
between the node and signer. Logs and metrics were exported for later analysis.

\section{Results and Discussion}
Preliminary results indicate that the Raspberry Pi platform is capable of
meeting the minimum requirements for BLS attestation signing. Median signing
latency remained within acceptable thresholds for timely inclusion. CPU
utilization increased under peak attestation aggregation but did not compromise
availability, and memory consumption remained stable throughout the observation
window.

These findings suggest that Pi-class hardware can serve as a viable environment
for prototyping threshold signatures and remote signer research. While
single-board devices are unlikely to replace industrial setups for high-value
mainnet operations, they provide an accessible platform for education, testing,
and iterative development, thereby broadening participation in consensus
research.

\section{Conclusion}
The Dalhousie Baker project demonstrates that BLS signing can be explored on
lightweight, accessible hardware without compromising the spirit of research and
participation. Future work includes multi-party key sharing experiments,
comparisons with industrial remote signing frameworks such as Signatory, and
the integration of longer-term monitoring to capture degradation or drift in
resource profiles over time.

\section*{Acknowledgments}
We thank the Tezos community for open discussions on aggregated attestations
(\href{https://forum.tezosagora.org/t/aggregated-attestations-a-heads-up-for-bakers/6770}{Agora thread}),
and our industrial cousin \href{https://dalekbaker.io}{DalekBaker.io} for
inspiration and methodological contrast.

\end{document}
